\documentclass{article}

\usepackage{amsmath}
\usepackage{amsfonts}
\usepackage{listings}

\begin{document}

The evaluation of the second deriviative is given by: $eval''(x) =
	derive \circ derive \circ eval (x)$.

Where $eval : FunExp \rightarrow FunSem $ and $derive : FunExp \rightarrow FunExp$. As a reminder,
the $FunExp$ type has the following definition:
\begin{lstlisting}
  data FunExp = Const REAL
              | X
              | FunExp :+: FunExp
              | FunExp :*: FunExp
              | Recip FunExp
              | Negate FunExp
              | Exp FunExp
              | Sin FunExp
              | Cos FunExp
\end{lstlisting}

And \lstinline|FunSem| has the following declaration:

\begin{lstlisting}
  type FunSem = REAL -> REAL
\end{lstlisting}

Consider the predicate:
\\
$P(h) = $\textit{"$h$ is a homomorphism from FunExp to FunSem"}

Or more formally, given the definition for \lstinline|FunExp| and \lstinline|FunSem|:

\[P (h) = H_{1}(h,\ Const,\ const)\]
\[\wedge\ H_{0}(h,\ X,\ id)\]
\[\wedge\ H_{2}(h,\ \text{:+:},\ +)\]
\[\wedge\ H_{2}(h,\ \text{:*:},\ *)\]
\[\wedge\ H_{1}(h,\ Recip,\ \lambda x.x^{-1})\]
\[\wedge\ H_{1}(h,\ Negate,\ \lambda x.-x)\]
\[\wedge\ H_{1}(h,\ Exp,\ \lambda x.e^{x})\]
\[\wedge\ H_{1}(h,\ Sin,\ sin)\]
\[\wedge\ H_{1}(h,\ Cos,\ cos)\]

Here the formal defintion for $H_{n \in \mathbf{N}}$ is assumed to be known as the
homomorphism predicate for operators of degree $n$.

\end{document}

